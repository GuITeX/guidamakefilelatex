\cleardoublepage{}
\chapter{Esempi d'uso}
\label{cha:esempi}

Il codice~\ref{lst:variabili} è un \texttt{Makefile} già abbastanza elaborato.
In realtà, come già osservato, tutte le operazioni che esso svolge possono
essere eseguite anche dai comuni editor di testo specializzati per \LaTeX{},
quindi ci si potrebbe chiedere, a ragione, perché scrivere un \texttt{Makefile}
piuttosto che affidarsi agli strumenti già disponibili.  \texttt{make} è utile
quando è necessario eseguire operazioni non comuni difficilmente eseguibili con
altri strumenti.  In questo paragrafo vedremo alcuni esempi di operazioni che un
\texttt{Makefile} permette di automatizzare.

\section{Convertire immagini \textsc{PDF} in formato \textsc{EPS}}
\label{sec:immagini-eps}

Quando in un documento si inseriscono delle immagini, queste devono avere un
formato particolare a seconda che si compili il documento con \LaTeX{}
o con \textsc{PDF}\LaTeX.  In particolare, \LaTeX{}
richiede immagini in formato \textsc{EPS}, \textsc{PDF}\LaTeX{}
accetta immagini in formato \textsc{PDF}, \textsc{JPG} e \textsc{PNG} (vedi
\cite[pagina 105]{pantieri:latex}).  Se si possiede un'immagine in un formato
diverso da quello necessario per l'inserimento nel documento, si deve quindi
procedere alla conversione da un formato all'altro.

Nelle distribuzioni GNU/Linux e con la distribuzione TeX Live è disponibile il
programma da linea di comando \texttt{pdftops}. Con la sintassi
\begin{sintassi}
  \small \texttt{\$ pdftops -eps} \meta{file PDF} \meta{file EPS}
\end{sintassi}
si converte il cioè \meta{file PDF} in formato \textsc{EPS}.  Il nome del nuovo
file è dedotto automaticamente dal file originale, cambiando solo l'estensione
da \texttt{.pdf} a \texttt{.eps}.  Se si vuole specificare un nome differente
per il file di output lo si può specificare come secondo argomento, opzionale,
del comando.

Supponiamo di avere nella sottocartella \texttt{Immagini} della cartella in cui
si trovano il documento \LaTeX{}
principale e il \texttt{Makefile} tutte le immagini in formato \textsc{PDF}.
Per compilare in \textsc{DVI} è quindi necessario convertire tutte le immagini.
Per automatizzare la compilazione e la conversione delle immagini è possibile
scrivere queste regole utilizzando le funzioni apprese nel
paragrafo~\ref{sec:variabili}:
\begin{lstlisting}[caption={\texttt{Makefile} in cui le immagini \textsc{PDF}
vengono convertite in \textsc{EPS} nella compilazione con \LaTeX.},label=lst:pdf-eps]
PRINCIPALE     = documento
PRINCIPALE_TEX = $(PRINCIPALE).tex
PRINCIPALE_DVI = $(PRINCIPALE).dvi
BIBLIOGRAFIA   = bibliografia.bib
TUTTI_LATEX    = $(PRINCIPALE_TEX) \
                 $(BIBLIOGRAFIA)
CARTELLA_IMG   = Immagini
IMMAGINI_PDF   = $(wildcard $(CARTELLA_IMG)/*.pdf)
IMMAGINI_EPS   = $(patsubst $(CARTELLA_IMG)/%.pdf,\
                 $(CARTELLA_IMG)/%.eps, \
		 $(IMMAGINI_PDF))

.PHONY: dvi

dvi: $(PRINCIPALE_DVI)

$(PRINCIPALE_DVI): $(TUTTI_LATEX) $(IMMAGINI_EPS)
        latex $(PRINCIPALE)
        bibtex $(PRINCIPALE)
        makeindex $(PRINCIPALE)
        latex $(PRINCIPALE)
        latex $(PRINCIPALE)

$(CARTELLA_IMG)/%.eps: $(CARTELLA_IMG)/%.pdf
	pdftops -eps $^ $@
\end{lstlisting} % $
La variabile \texttt{IMMAGINI\_PDF} contiene l'elenco di tutti i file in formato
\textsc{PDF} presenti nella sottocartella \texttt{Immagini} (il nome della
sottocartella è salvato nella variabile \texttt{CARTELLA\_IMG} in modo che sarà
sufficiente cambiare solo questo valore per cartelle con nomi differenti).  La
variabile \texttt{IMMAGINI\_EPS}, invece, contiene l'elenco dei degli stessi
file in formato \textsc{PDF}, ma questa volta con estensione \texttt{.eps}
(abbiamo utilizzato la funzione \texttt{patsubst} per eseguire la sostituzione).
In questo \texttt{Makefile} è presente una regola con una definizione un po'
particolare:
\begin{lstlisting}
$(CARTELLA_IMG)/%.eps: $(CARTELLA_IMG)/%.pdf
	pdftops -eps $^ $@
\end{lstlisting} %$
Il prerequisito di questa regola è un qualsiasi file della cartella
\texttt{Immagini} con estensione \texttt{.pdf}, l'obiettivo, però, non è un
qualsiasi file della sottocartella \texttt{Immagini} con estensione
\texttt{.eps}, bensì il file che ha lo stesso nome base del prerequisito ed
estensione
\texttt{.eps}.\footnote{Il metacarattere \texttt{\%} è stato introdotto nel
  paragrafo~\ref{sec:variabili}.}
Con questa regola verranno dunque generati solo file con estensione
\texttt{.eps} e nome base uguale a quello di file in formato \textsc{PDF}
presente nella sottocartella \texttt{Immagini}.  Il comando eseguito nella
regola è
\begin{lstlisting}
	pdftops -eps $^ $@
\end{lstlisting}
Le due variabili \texttt{\$\^} e \texttt{\$@} sono delle variabili dette
\emph{automatiche}\footnote{Per un elenco esaustivo di queste variabili vedi
  \cite[pagina 112]{gnu:make}.}
e che hanno un significato speciale: \texttt{\$\^} indica tutti i prerequisiti
della regola (in questo caso indica il solo file \textsc{PDF} che ha lo stesso
nome di base dell'obiettivo), la seconda indica l'obiettivo della regola (in
questo caso l'immagine \textsc{EPS} da generare).  Un'altra variabile automatica
utile è \texttt{\$<} che è uguale al primo prerequisito.  Se volessimo
convertire un determinato file chiamato, supponiamo, \texttt{immagine.pdf} in
formato \textsc{EPS} potremmo utilizzare il comando nel terminale:
\begin{verbatim}
$ make immagine.eps
\end{verbatim}
% $
Per come è scritta la regola con obiettivo \texttt{\$(PRINCIPALE\_DVI)} nel
\texttt{Makefile} del codice~\ref{lst:pdf-eps} non è però necessario convertire
una a una le immagini quando si vuole generare il documento in formato
\textsc{DVI}.  Infatti, compilando con il comando
\begin{verbatim}
$ make dvi
\end{verbatim}
% $
il programma \texttt{make} si preoccupa di verificare se tutti i prerequisiti
elencati in questa regola sono aggiornati.  Fra questi ci sono le figure
\texttt{\$(IMMAGINI\_EPS)} che verranno generate a partire dalle corrispondenti
immagini in formato \textsc{PDF} grazie alla regola vista in precedenza.

% $
Immaginiamo di avere in una cartella il file \texttt{documento.tex} come file
\LaTeX{}
principale, \texttt{bibliografia.bib} come raccolta dei riferimenti
bibliografici.  Nella sottocartella \texttt{Capitoli} sono presenti vari file
\texttt{.tex} inclusi in \texttt{documento.tex}; nella sottocartella
\texttt{Immagini}, invece, sono presenti immagini nel solo formato \textsc{PDF}
che saranno eventualmente convertite nel formato \textsc{EPS} nel caso di
compilazione con \LaTeX.  Un \texttt{Makefile} per compilare questo progetto
potrebbe apparire così:
\begin{lstlisting}[caption={\texttt{Makefile} in cui la prima regola compila il
documento in \textsc{DVI} convertendo le immagini \textsc{PDF} in \textsc{EPS},
la seconda regola compila in formato \textsc{PDF}.},
label=lst:dvi+pdf]
PRINCIPALE 	= documento
PRINCIPALE_TEX	= $(PRINCIPALE).tex
PRINCIPALE_DVI	= $(PRINCIPALE).dvi
PRINCIPALE_PDF	= $(PRINCIPALE).pdf
CAPITOLI_TEX	= $(wildcard Capitoli/*.tex)
BIBLIOGRAFIA	= bibliografia.bib
TUTTI_LATEX	= $(PRINCIPALE_TEX) \
		  $(BIBLIOGRAFIA) \
		  $(CAPITOLI_TEX)
IMMAGINI_PDF	= $(wildcard Immagini/*.pdf)
IMMAGINI_EPS	= $(patsubst Immagini/%.eps,\
		  Immagini/%.pdf, $(IMMAGINI_PDF))
FILE_CLEAN	= *.aux *.bbl *.blg *.brf *.idx \
		  *.ilg *.ind *.log
FILE_DISTCLEAN	= $(PRINCIPALE_DVI) \
		  $(PRINCIPALE_PDF) \
		  $(IMMAGINI_EPS)

.PHONY: dvi pdf distclean clean

dvi: $(PRINCIPALE_DVI)

pdf: $(PRINCIPALE_PDF)

$(PRINCIPALE_DVI): $(TUTTI_LATEX) $(IMMAGINI_EPS)
	latex $(PRINCIPALE)
	bibtex $(PRINCIPALE)
	makeindex $(PRINCIPALE)
	latex $(PRINCIPALE)
	latex $(PRINCIPALE)

$(PRINCIPALE_PDF): $(TUTTI_LATEX) $(IMMAGINI_PDF)
	pdflatex $(PRINCIPALE)
	bibtex $(PRINCIPALE)
	makeindex $(PRINCIPALE)
	pdflatex $(PRINCIPALE)
	pdflatex $(PRINCIPALE)

Immagini/%.eps: Immagini/%.pdf
	pdftops -eps $^ $@

distclean: clean
	rm -f $(FILE_DISTCLEAN)

clean:
	rm -f $(FILE_CLEAN)
\end{lstlisting}
\begin{figure}
  \centering
  \begin{tikzpicture}[level 1/.style={sibling distance=90},
    edge from parent/.style={draw,latex-},font=\small]
    \node (dvi) [rounded corners,draw] {\texttt{\$(PRINCIPALE\_DVI)}}
      child {node {\texttt{\$(TUTTI\_LATEX)}}}
      child {node[rounded corners,draw] {\texttt{\$(IMMAGINI\_EPS)}}
        child {node {\texttt{\$(IMMAGINI\_PDF)}}}
      };
    \node at ($(dvi)+(6.2,0)$) [rounded corners,draw]
      {\texttt{\$(PRINCIPALE\_PDF)}}
      child {node {\texttt{\$(TUTTI\_LATEX)}}}
      child {node {\texttt{\$(IMMAGINI\_PDF)}}};
  \end{tikzpicture}
  \caption{Grafo ad albero rappresentante le dipendenze fra i file del
    \texttt{Makefile} contenuto nel codice~\ref{lst:dvi+pdf}, eccetto i phony
    targets.}
  \label{fig:grafo-albero2}
\end{figure}
Con gli strumenti acquisiti fino a questo punto dovrebbe essere comprensibile
come funziona questo \texttt{Makefile}, cosa compie ciascuna regola e quali
comandi bisogna dare nel terminale per compilare il documento.  Per maggiore
chiarezza, nella figura~\ref{fig:grafo-albero2} è rappresentato il grafo ad
albero che descrive le dipendenze fra i file del \texttt{Makefile} mostrato nel
codice~\ref{lst:dvi+pdf}.

\section{Creare un archivio compresso}
\label{sec:creare-archivio}

Spesso si vuole creare un archivio compresso contenente il codice sorgente del
proprio documento, in modo da poterlo distribuire in maniera semplice.  Questo
può essere fatto con un codice come quello seguente
\begin{lstlisting}
CARTELLA	= $(shell basename $$(pwd))
PRINCIPALE	= documento
DATA		= $(shell date "+%Y%m%d-%H%M%S")

.PHONY: dist distclean

# Questo Makefile è incompleto, per la definizione del
# phony target `distclean' vedi gli esempi precedenti

dist: distclean
	cd ..; tar -cvaf $(PRINCIPALE)-$(DATA).tar.xz \
	  --exclude-vcs $(CARTELLA)/
\end{lstlisting} % $
La variabile \texttt{CARTELLA} ha come valore il nome della cartella corrente:
nella shell il percorso della cartella corrente si ottiene con il comando
\texttt{pwd}, il suo nome base si estrae con \texttt{basename \$(pwd)}.  La
sostituzione di comando con \texttt{\$\$} è stata spiegata nel
paragrafo~\ref{sec:sostituzione-comando}, la funzione \texttt{\$(shell ...)} è
stata introdotta nel paragrafo~\ref{sec:uso-variabili}.  Il phony target
\texttt{dist} esegue queste operazione: cambia la cartella attuale di lavoro
nella cartella superiore (con il comando \texttt{cd ..}) e crea in questa
posizione un archivio, compresso con il programma \texttt{xz}, della cartella
\texttt{\$(CARTELLA)}.  Il nome dell'archivio è
\texttt{\$(PRINCIPALE)-\$(DATA).tar.xz}, in modo da individuare facilmente la
data in cui è stato creato, e con l'opzione \texttt{--exclude-vcs} si escludono
dall'archivio gli eventuali file legati al sistema di controllo delle revisioni
usato.  Per maggiori informazioni sulle opzioni di \texttt{tar} si rimanda al
suo manuale consultabile con il comando
\begin{verbatim}
$ man tar
\end{verbatim}
% $

Si noti che nell'elenco dei comandi \texttt{cd} e \texttt{tar} devono trovarsi
sulla stessa riga, separati da un punto e virgola \texttt{;} per separarli, e
non uno per riga come ci si potrebbe aspettare, perché, come detto nel
paragrafo~\ref{sec:le-regole}, \texttt{make} in maniera predefinita invoca una
nuova subshell per ogni riga dell'elenco dei comandi.  Quindi se si scrivesse
\begin{lstlisting}
dist: distclean
	cd ..
	tar -cvaf $(PRINCIPALE)-$(DATA).tar.xz \
	  --exclude-vcs $(CARTELLA)/
\end{lstlisting}
% $
il comando \texttt{tar} sarebbe eseguito nella stessa cartella del codice
sorgente e non nella cartella superiore, come vorremmo.

In questo esempio il phony target \texttt{distclean}, già mostrata negli esempi
precedenti, è un prerequisito di \texttt{dist}, quindi eseguendo da terminale il
comando
\begin{verbatim}
$ make dist
\end{verbatim}
% $
verranno prima cancellati tutti i file generati dalla compilazione del documento
e poi verrà creato l'archivio, che in questo conterrà esclusivamente il codice
sorgente.  Se invece si vuole che l'archivio contenga, per esempio, il file di
output in formato \textsc{PDF} ma non tutti i file temporanei generati, i
prerequisiti della regola \texttt{dist} possono essere cambiati in questo modo
\begin{lstlisting}
dist: pdf clean
\end{lstlisting}
I phony targets \texttt{pdf} e \texttt{clean} sono gli stessi già illustrati in
precedenza e devono essere elencati fra i prerequisiti nel preciso ordine
riportato qui, in modo che prima venga generato il \textsc{PDF} e
successivamente vengano cancellati i file temporanei.

\section{Includere immagini esterne da compilare}
\label{sec:includere-immagini-esterne}

Ti\emph{k}Z è un potente pacchetto \LaTeX{}
per la realizzazione di disegni.  Se in un documento si inseriscono numerosi
disegni realizzati con Ti\emph{k}Z e anche molto elaborati, i tempi di
compilazione potrebbero aumentare sensibilmente.  Una soluzione a questo
problema potrebbe essere la seguente: mettere il codice di ciascuna figura
Ti\emph{k}Z in un file con estensione \texttt{.tikz} usando la classe
\texttt{standalone}, compilare tutti questi file con \textsc{PDF}\LaTeX{}
per ottenere il documento in formato \textsc{PDF} e poi includere i file così
prodotti nel documento principale con la macro
\texttt{\textbackslash{}includegraphics\{\}}.  In questa situazione diventa
importante assicurarsi che le figure \textsc{PDF} siano aggiornate, infatti è
facile dimenticarsi di ricompilare la figura dopo aver fatto una piccola
modifica al suo codice sorgente.  È evidente che in questo caso \texttt{make}
può risultare molto utile per controllare in maniera efficiente che le figure
Ti\emph{k}Z siano sempre aggiornate.

Supponiamo che tutti i file \texttt{*.tikz} siano posti nella sottocartella
\texttt{img-tikz} della directory in cui si trovano il \texttt{Makefile} e il
file principale del documento, chiamato \texttt{documento.tex}. Per i codici
delle figure si potrebbe più semplicemente usare la classica estensione
\texttt{.tex}, si è preferito \texttt{.tikz} per evidenziare che si tratta di
codici di figure Ti\emph{k}Z.  Inoltre questa scelta permette di riservare
l'estensione \texttt{.tex} a un eventuale file, da non compilare, posto nella
cartella \texttt{img-tikz} e contenente solo il preambolo comune per tutti i
file \texttt{*.tikz}, all'interno dei quali viene incluso con la macro
\texttt{\textbackslash{}input\{\}}.  Nell'esempio seguente il file di preambolo
è chiamato \texttt{preambolo.tex}.

Un \texttt{Makefile} che permette di compilare il file \texttt{documento.tex} in
formato \textsc{PDF} è il seguente
\begin{lstlisting}
PRINCIPALE 	    = documento
PRINCIPALE_TEX	    = $(PRINCIPALE).tex
PRINCIPALE_PDF	    = $(PRINCIPALE).pdf
CAPITOLI_TEX	    = $(wildcard Capitoli/*.tex)
TUTTI_LATEX	    = $(PRINCIPALE_TEX) \
		      $(CAPITOLI_TEX)
IMMAGINI_TIKZ	    = $(wildcard img-tikz/*.tikz)
IMMAGINI_TIKZ_PDF   = $(patsubst img-tikz/%.tikz,\
		      img-tikz/%.pdf, $(IMMAGINI_TIKZ))
FILE_CLEAN	    = *.aux *.log \
		      $(wildcard img-tikz/*.aux) \
		      $(wildcard img-tikz/*.log)
FILE_DISTCLEAN	    = $(PRINCIPALE_PDF) \
		      $(IMMAGINI_TIKZ_PDF)

.PHONY: pdf distclean clean

pdf: $(PRINCIPALE_PDF)

# Supponiamo per brevità che per compilare il
# documento principale sia sufficiente eseguire
# due volte pdflatex
$(PRINCIPALE_PDF): $(TUTTI_LATEX) $(IMMAGINI_TIKZ_PDF)
	pdflatex $(PRINCIPALE)
	pdflatex $(PRINCIPALE)

img-tikz/%.pdf: img-tikz/%.tikz img-tikz/prambolo.tex
	cd img-tikz ; pdflatex $(shell basename $<)

distclean: clean
	rm -f $(FILE_DISTCLEAN)

clean:
	rm -f $(FILE_CLEAN)
\end{lstlisting}
La duplice utilità di un simile \texttt{Makefile} è che le figure \textsc{PDF}
realizzate con Ti\emph{k}Z sono sicuramente sempre aggiornate quando si genera
il documento finale con \texttt{make} ed esse vengono ricompilate solo se il
corrispondente codice sorgente è stato modificato, riducendo al minimo
indispensabile il tempo di compilazione totale.

%%% Local Variables:
%%% mode: latex
%%% TeX-master: "../guidamake"
%%% End:
