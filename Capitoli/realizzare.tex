\cleardoublepage
\chapter{Realizzare un \texttt{Makefile}}
\label{cha:realizzare-makefile}

\section{Le regole}
\label{sec:le-regole}

Un \verb|Makefile| basilare è composto essenzialmente da \emph{regole} (chiamate
in inglese \emph{rules}) che hanno la seguente struttura (vedi
\textcite[3]{gnu:make}):
\begin{lstlisting}[showtabs=true,tab=\rightarrowfill,caption={Struttura di una
regola},label=lst:rule]
obiettivo: prerequisiti
	comandi
	...
	...
\end{lstlisting}
L'\emph{obiettivo} (in inglese \emph{target}), ciò che viene scritto prima dei
due punti \verb|:|, di norma è il nome del file che verrà generato con la regola
descritta.  Vedremo nel paragrafo~\ref{sec:phony} in che modo sia possibile
usare un obiettivo che non sia il nome di un file da creare eseguendo la
corrispondente regola.

I \emph{prerequisiti} sono i file che devono essere presenti al momento
dell'esecuzione di una regola e a partire dai quali viene generato il file
obiettivo.  Spesso gli obiettivi dipendono da più file, che vengono elencati
separati da uno spazio.  Per evitare problemi in fase di compilazione è
conveniente che tutti i file non abbiano degli spazi nel proprio nome.  Le
regole possono non avere alcun prerequisito.

Il programma \verb|make| conosce i comandi necessari per eseguire una regola
leggendoli dall'elenco dei \emph{comandi}.  I comandi di ciascuna regola
\emph{devono necessariamente} essere preceduti da una tabulazione (nel
codice~\ref{lst:rule} è evidenziata da una lunga linea), non da spazi altrimenti
verrà segnalato un errore.

\begin{figure}
  \centering
  % TODO: provare a usare le funzioni native di TikZ per creare grafi ad albero
  \begin{tikzpicture}
    \node (p1) at (0,0) {\verb|prerequisito1|};
    \node[rounded corners,draw] (p2) at (4,0) {\verb|prerequisito2|};
    \node (p3) at ($(p2)-(0,1)$) {\verb|prerequisito3|};
    \node[rounded corners,draw] (o1) at ($(p1)!0.5!(p2)+(0,1)$)
                                {\verb|obiettivo1|};
    \draw[->] (p1) -- (o1);
    \draw[->] (p2) -- (o1);
    \draw[->] (p3) -- (p2);
  \end{tikzpicture}
  \caption{Grafo ad albero che illustra le dipendenze fra i file.  I file che si
    trovano alla coda di una freccia, evidenziati in un rettangolo,
    costituiscono l'obiettivo di una regola, i file che invece si trovano alla
    testa di una freccia rappresentano i prerequisiti della corrispondente
    regola e sono necessari per la creazione dell'obiettivo.}
  \label{fig:grafo-albero1}
\end{figure}
Per scrivere correttamente un \verb|Makefile| risulta utile disegnarsi un grafo
ad albero che illustri le relazioni che ci sono fra i vari file, individuando
quali dovranno essere gli obiettivi delle regole e quali saranno i prerequisiti
delle regole.  Uno stesso file può svolgere il ruolo di prerequisito in una
regola e di obiettivo in un'altra.  Nell'esempio della
figura~\ref{fig:grafo-albero1} ci sono due obiettivi, \verb|obiettivo1| e
\verb|prerequisito2|, per le quali verranno scritte le opportune regole.  I
prerequisiti di \verb|obiettivo1| sono i file \verb|prerequisito1| e
\verb|prerequisito2|, l'unico prerequisito di \verb|prerequisito2| è il file
\verb|prerequisito3|.

Un \verb|Makefile| deve contenere almeno una regola, ma possono essercene anche
più di una, ciascuna corrispondente a un diverso file da creare, e l'ordine con
cui le regole compaiono nel \verb|Makefile| \emph{non} è importante.

Un \verb|Makefile| può contenere, oltre alle regole, anche dei commenti,
introdotti dal simbolo \verb|#|: tutto ciò che si trova alla destra di \verb|#|
viene trattato come commento.  Questo simbolo ha lo stesso comportamento del
simbolo \verb|%| nel linguaggio \LaTeX{}, vedi \textcite[26]{pantieri:latex}.

Se lo si desidera è possibile suddividere una riga di codice molto lunga su più
righe inserendo \verb|\| seguito da un carattere di nuova linea (in pratica
bisogna premere il tasto \keystroke{Invio} subito dopo l'inserimento del
backslash).  Ciò non è comunque obbligatorio, in quanto non sono posti limiti
alla lunghezza delle righe di un \verb|Makefile|.


\section{Come funziona \texttt{make}}
\label{sec:come-funziona}

Per eseguire la regola che ha per obiettivo il file \verb|<obiettivo>| bisogna
invocare \verb|make| da terminale con il comando
\begin{verbatim}
make <obiettivo>
\end{verbatim}
\verb|make| verifica se è necessario aggiornare l'obiettivo della regola
corrispondente in questo modo: se l'obiettivo indicato da linea di comando non
esiste oppure ha una data di ultima modifica precedente almeno a una di quelle
dei file elencati nei \emph{prerequisiti} è considerato da aggiornare e la
regola viene eseguita.  In caso contrario (l'obiettivo esiste ed è più recente
di tutti i prerequisiti) la regola non viene eseguita e sul terminale si leggerà
il messaggio
\begin{verbatim}
make: Nessuna operazione da eseguire per "<obiettivo>".
\end{verbatim}
Le istruzioni per aggiornare un obiettivo sono
presenti nei \emph{comandi}: questi sono passati alla shell, normalmente la
\verb|sh|.

Come detto, uno dei prerequisiti di una regola può essere l'obiettivo di
un'altra regola.  In questo caso \verb|make| verifica se, prima di eseguire una
regola, è necessario aggiornare uno o più dei file elencati fra i prerequisiti.
Nel seguente \verb|Makefile|, corrispondente alla situazione illustrata nella
figura~\ref{fig:grafo-albero1},
\begin{lstlisting}
obiettivo1: prerequisito1 prerequisito2
	comando1
	comando2

prerequisito2: prerequisito3
	comando3
\end{lstlisting}
il \verb|prerequisito2| ha una propria regola: dando il comando
\begin{verbatim}
make obiettivo1
\end{verbatim}
\verb|make| verifica se è necessario aggiornare, eseguendo il \verb|comando3|
specificato, il file \verb|prerequisito2| prima di eseguire la regola associata
all'\verb|obiettivo1|.

Tutta la logica di funzionamento di \verb|make| si basa sul fatto che gli
obiettivi dipendono dai prerequisiti:
\emph{quando un prerequisito viene modificato allora probabilmente l'obiettivo
  deve essere ricreato}.
Questo meccanismo, forse un po' laborioso e difficile da comprendere
inizialmente, dovrebbe risultare chiaro più avanti.

Se si richiama \verb|make| senza argomenti, cioè si dà solo il comando
\begin{verbatim}
make
\end{verbatim}
esso procede all'esecuzione della prima regola che trova nel \verb|Makefile|,
per la precisione la prima regola il cui target non inizia con un punto \verb|.|
(questo comportamento può essere modificato, vedi \textcite[5]{gnu:make}).

Se si vuole forzare l'esecuzione delle regola \verb|<obiettivo>|, ignorando le
date di modifica dell'obiettivo e dei suoi prerequisiti, bisogna utilizzare
l'opzione \texttt{-B} in questo modo:
\begin{verbatim}
make -B <obiettivo>
\end{verbatim}

\section{Un semplice \texttt{Makefile}}
\label{sec:makefile-semplice}

Passiamo ora alla pratica vedendo un esempio di \verb|Makefile| molto semplice
per compilare un documento \LaTeX{}:
\begin{lstlisting}[caption={Un semplice \texttt{Makefile}},label=lst:base,
showtabs=true,tab=\rightarrowfill]
documento.dvi: documento.tex
	latex documento
\end{lstlisting}
Nel codice~\ref{lst:base}, il file \verb|documento.dvi| è l'\emph{obiettivo},
cioè il file ottenuto dopo la compilazione, il \emph{prerequisito} è il solo
file \verb|documento.tex| e l'unico comando che deve essere eseguito è
\verb|latex documento|.

Per invocare \verb|make|, dopo aver aperto un terminale ed essersi spostati con
il comando \verb|cd| nella cartella in cui si trova il \verb|Makefile|, bisogna
utilizzare il comando
\begin{verbatim}
make obiettivo
\end{verbatim}
sostituendo a \verb|obiettivo| il nome dell'obiettivo che specificato nella
regola che si desidera eseguire.  In questo caso dovremo eseguire il comando
\begin{verbatim}
make documento.dvi
\end{verbatim}
oppure semplicemente \verb|make| se la regola è la prima in ordine di
apparizione nel \verb|Makefile|.


\section{Un \texttt{Makefile} un po' più elaborato}
\label{sec:makefile-elaborato}

Se nel nostro file di esempio \verb|documento.tex| è presente una bibliografia
realizzata con \textsc{Bib}\TeX, per compilare il documento è necessario (vedi
\textcite[120]{pantieri:latex}) eseguire i comandi
\begin{verbatim}
latex documento
bibtex documento
latex documento
latex documento
\end{verbatim}

Ripetere questi cinque comandi ogni volta che si desidera compilare un documento
può diventare, quindi, un'operazione noiosa.  È in questi casi che si vede
l'utilità di \verb|make|.  Nel nostro \verb|Makefile| possiamo mettere la
seguente regola
\begin{lstlisting}
documento.dvi: documento.tex bibliografia.bib
	latex documento
	bibtex documento
	latex documento
	latex documento
\end{lstlisting}
in cui i prerequisiti (cioè i file che devono essere presenti e aggiornati) sono
il file principale \verb|documento.tex| e il file in cui abbiamo scritto la
nostra base di dati dei riferimenti bibliografici.  Per compilare è sufficiente
dare il comando
\begin{verbatim}
make documento.dvi
\end{verbatim}
oppure solo \verb|make| qualora quella regola fosse la prima presente nel
\verb|Makefile|.

Si può anche mettere una regola per compilare lo stesso documento con \LaTeX{}
e un'altra per compilarlo con \textsc{PDF}\LaTeX, per poter scegliere fra un
file di output in formato \textsc{DVI} o \textsc{PDF}.  In questo caso il
\verb|Makefile| apparirebbe così:
\begin{lstlisting}[caption={La prima regola permette di compilare un documento con
\LaTeX, la seconda con \textsc{PDF}\LaTeX},label=lst:dvi-pdf]
documento.dvi: documento.tex bibliografia.bib
	latex documento
	bibtex documento
	latex documento
	latex documento

documento.pdf: documento.tex bibliografia.bib
	pdflatex documento
	bibtex documento
	pdflatex documento
	pdflatex documento
\end{lstlisting}

In alcune distribuzioni GNU/Linux sono presenti gli script \verb|texi2dvi| e
\verb|texi2pdf| che eseguono rispettivamente \LaTeX{}
e \textsc{PDF}\TeX{}
(e, se necessario, \textsc{Bib}\TeX) sul sorgente il numero strettamente
necessario di volte per la corretta compilazione del documento (vedi
\textcite[63]{caucci:tabelle}).  Utilizzando questi due comandi il
codice~\ref{lst:dvi-pdf} si potrebbe ridurrebbe quindi a
\begin{lstlisting}
documento.dvi: documento.tex bibliografia.bib
	texi2dvi documento

documento.pdf: documento.tex bibliografia.bib
	texi2pdf documento
\end{lstlisting}
Anche lo script \verb|latexmk|, fornito dalle distribuzioni TeX Live e MikTeX,
offre funzionalità simili, bisogna però evidenziare che se il documento richiede
comandi particolari per la compilazione (come quando si utilizza il pacchetto
\verb|frontespizio|), \verb|texi2dvi|, \verb|texi2pdf| e \verb|latexmk| non sono
più in grado di generare correttamente il documento finale.

\section{Phony target}
\label{sec:phony}

È possibile specificare delle regole che non hanno come obiettivo il nome del
file che verrà ottenuto.  Questo tipo di obiettivi vengono chiamati in inglese
\emph{phony targets} (cioè \emph{falsi obiettivi}).  Spesso i phony target hanno
come nome dei comandi. Per esempio, quando si esegue la regola
\begin{lstlisting}
clean:
	rm -f *.aux *.log *.out
\end{lstlisting}
vengono cancellati tutti i file con estensione \verb|.aux|, \verb|.log| e
\verb|.out| che vengono prodotti durante la compilazione di un semplice
documento
\LaTeX.\footnote{Il comando \texttt{rm} cancella tutti i file che vengono
  elencati di seguito, l'opzione \texttt{-f} serve per non stampare eventuali
  messaggi di errore in caso di assenza dei file da cancellare.  Il
  metacarattere \texttt{*} sostituisce una qualsiasi sequenza di caratteri.}
È consigliabile specificare esplicitamente quali sono i phony target utilizzati
nel \verb|Makefile|: se nella cartella in cui si trova il \verb|Makefile| è
presente un file chiamato \verb|clean| questa regola non verrebbe mai eseguita.
Infatti, dal momento che la regola non ha prerequisiti, il file \texttt{clean}
risulterebbe sempre aggiornato (vedi \textcite[31]{gnu:make}).  Per fare ciò
bisogna mettere gli obbiettivi di queste regole come prerequisiti della regola
speciale \verb|.PHONY|:
\begin{lstlisting}
.PHONY: clean
clean:
	rm -f *.aux *.log *.out
\end{lstlisting}
In questo modo \verb|make| sa che \verb|clean| non è il nome del file che si
deve ottenere e la regola verrà quindi sempre eseguita, indipendentemente dalla
presenza di un eventuale file \verb|clean|.

I phony target possono essere anche usati per creare una sorta di \emph{alias}
di altre regole.  Per esempio, inserendo il seguente
\begin{lstlisting}[caption=I prerequisiti della regola dell'obiettivo \texttt{.PHONY}
sono i nomi dei phony target che vengono successivamente specificati,label=lst:phony]
.PHONY: dvi pdf

dvi: documento.dvi

pdf: documento.pdf
\end{lstlisting}
in un \verb|Makefile|, prima del codice~\ref{lst:dvi-pdf}, per compilare il
documento in formato \textsc{DVI} si potrà eseguire il comando
\begin{verbatim}
make dvi
\end{verbatim}
Analogamente, per ottenere il file \textsc{PDF} si potrà dare il comando
\begin{verbatim}
make pdf
\end{verbatim}
L'utilità dell'uso di questi alias è che i comandi da eseguire sono indipendenti
dal nome del file di output.

Accanto al phony target \verb|clean| si trova spesso \verb|distclean|:
\verb|clean| cancella solo i file temporanei generati durante la compilazione,
\verb|distclean| in più elimina anche i file di output (quindi gli eventuali
file in formato \verb|.pdf| o \verb|.dvi|, nel caso di documenti
\LaTeX).\footnote{Queste sono solo delle convenzioni, in uso specialmente
  nell'ambito della programmazione.  L'utente è libero di creare regole diverse
  e con nomi differenti.}
Poiché \verb|distclean|, \emph{oltre} a cancellare file rimossi da \verb|clean|
ne cancella altri, è possibile inserire \verb|clean| come prerequisito di
\verb|distclean|, in modo che quella regola venga eseguita \emph{anche} quando
si dà il comando \verb|make distclean|:
\begin{lstlisting}[caption={Phony target \texttt{distclean} e \texttt{clean}.},
label=lst:distclean]
.PHONY: distclean clean

distclean: clean
	rm -f *.pdf *.dvi

clean:
	rm -f *.aux *.log *.out
\end{lstlisting}


\section{Le variabili}
\label{sec:variabili}

Uno dei punti di forza dell'utilizzo di \verb|make| per compilare documenti
\LaTeX{}
è che una volta che si possiede un \verb|Makefile| ben organizzato per compilare
un documento, con pochissime modifiche si può adattare alla compilazione di un
altro documento, strutturato in maniera simile.  Ciò è reso ancora più facile
dall'uso delle variabili.

Una variabile è un nome a cui è associato un \emph{valore} che in genere è una
stringa di testo.  Per assegnare a una variabile il suo valore si usa la
sintassi
\begin{lstlisting}
VARIABILE = valore
\end{lstlisting}
Le variabili permettono di rendere il \verb|Makefile| più compatto perché i
valori delle variabili sono spesso dei lunghi elenchi di file che dovrebbero
essere ripetuti più volte all'interno del file: invece di scrivere ogni volta
questa lunga stringa è sufficiente richiamare il valore della variabile che sarà
poi automaticamente sostituito da \verb|make| durante la processazione del file.
Inoltre quando diventa necessario modificare uno di questi elenchi, è
sufficiente modificare solo una volta il valore della variabile, senza dover
andare a cercare nel file tutte le occorrenze da sostituire.

Le variabili, \emph{dopo} essere state dichiarate, possono essere referenziate
usando il simbolo del dollaro seguito (senza spazi) da parantesi tonde o graffe:
\begin{lstlisting}
$(VARIABILE)
${VARIABILE}
\end{lstlisting}
Per evitare di dimenticarsi di dichiarare una variabile prima di richiamarla può
essere utile abituarsi a definire tutte le variabili all'inizio del
\verb|Makefile|.

Le variabili possono essere referenziate in qualsiasi parte di un
\verb|Makefile|, come per esempio negli obiettivi, nei prerequisiti, nei
comandi, nel valore di altre variabili.  Le variabili possono rappresentare
qualsiasi cosa: oltre a elenchi di file le variabili potrebbero avere come
valore nomi di cartelle in cui cercare file o programmi da eseguire.

Le variabili, come qualunque altra cosa scritta nel \verb|Makefile|, sono
sensibili alle maiuscole, quindi \verb|Variabile|, \verb|variabile| e
\verb|VARIABILE| sono stringhe distinte.  Inoltre il nome di una variabile può
essere una sequenza di qualsiasi carattere eccetto spazi o tabulazioni, siano
essi iniziali o finali, o uno fra i tre seguenti simboli \verb|:| \verb|#|
\verb|=|.  È comunque consigliabile utilizzare per i nomi delle variabili solo
lettere, numeri e trattini bassi, vedi \textcite[57]{gnu:make}.  Eventuali
caratteri di spaziatura o tabulazione presenti prima o dopo il nome di una
variabile vengono ignorati, come nel codice~\ref{lst:variabili}.

Vediamo ora un'applicazione dell'uso delle variabili.  Consideriamo il caso in
cui abbiamo un documento \LaTeX{}
principale chiamato \verb|documento.tex|, nel quale abbiamo inserito un indice
analitico e una bibliografia realizzata con \textsc{Bib}\TeX{}
e che l'elenco dei libri consultati si trovi nel file \verb|bibliografia.bib|.
Entrambi i file \verb|documento.tex| e \verb|bibliografia.bib|, inoltre, si
trovano nella stessa cartella in cui è posizionato il seguente
\verb|Makefile|:\footnote{Parte del \texttt{Makefile} è ripreso da quello
  presente in \textcite[61]{caucci:tabelle}.}
\begin{lstlisting}[caption={Esempio di \texttt{Makefile} che utilizza le
variabili},label=lst:variabili]
PRINCIPALE 		= documento
PRINCIPALE_TEX		= $(PRINCIPALE).tex
PRINCIPALE_DVI		= $(PRINCIPALE).dvi
PRINCIPALE_PDF		= $(PRINCIPALE).pdf
BIBLIOGRAFIA		= bibliografia
BIBLIOGRAFIA_BIB	= $(BIBLIOGRAFIA).bib
FILE_CLEAN		= *.aux *.bbl *.blg *.brf *.idx \
			  *.ilg *.ind *.log
FILE_DISTCLEAN		= $(PRINCIPALE_DVI) \
			  $(PRINCIPALE_PDF)

.PHONY: dvi pdf distclean clean

dvi: $(PRINCIPALE_DVI)

pdf: $(PRINCIPALE_PDF)

$(PRINCIPALE_DVI): $(PRINCIPALE_TEX) $(BIBLIOGRAFIA_BIB)
	latex $(PRINCIPALE)
	bibtex $(PRINCIPALE)
	makeindex $(PRINCIPALE)
	latex $(PRINCIPALE)
	latex $(PRINCIPALE)

$(PRINCIPALE_PDF): $(PRINCIPALE_TEX) $(BIBLIOGRAFIA_BIB)
	pdflatex $(PRINCIPALE)
	bibtex $(PRINCIPALE)
	makeindex $(PRINCIPALE)
	pdflatex $(PRINCIPALE)
	pdflatex $(PRINCIPALE)

distclean: clean
	rm -f $(FILE_DISTCLEAN)

clean:
	rm -f $(FILE_CLEAN)
\end{lstlisting}
Le variabili specificate all'inizio del file vengono sostituite da \verb|make|
quando viene invocato: tutte le occorrenze di
\verb|$(PRINCIPALE)| verranno lette dal programma come se ci fosse scritto
\verb|documento|, perciò la variabile \verb|$(PRINCIPALE_TEX)|
assume il valore \verb|documento.tex|, e così via. Come nell'esempio del
codice~\ref{lst:phony}, per compilare il documento in formato \textsc{DVI} è
sufficiente dare il comando
\begin{verbatim}
make dvi
\end{verbatim}
e per ottenere un documento \textsc{PDF} bisogna invece utilizzare il comando
\begin{verbatim}
make pdf
\end{verbatim}

Nella variabile \verb|$(FILE_CLEAN)|
abbiamo indicato tutti i file che dovranno essere cancellati nella regole
\verb|clean|, analogamente la variabile \verb|FILE_DISTCLEAN| assume come valore
i nomi dei file \verb|documento.dvi| e \verb|documento.pdf| che verranno rimossi
se si esegue il comando \verb|make distclean|.  Notare che, come nel
codice~\ref{lst:distclean}, \verb|clean| è un prerequisito di \verb|distclean|.

Quando si dovrà compilare un documento \LaTeX{}
che richiede gli stessi comandi del documento appena visto, si potrà facilmente
riutilizzare lo stesso \verb|Makefile|, avendo solo cura di modificare il valore
delle variabili \verb|PRINCIPALE| e \verb|BIBLIOGRAFIA| a seconda delle
necessità. % Il \verb|Makefile| va sempre posto o copiato nella cartella in cui
% si trova il nuovo documento.

Esistono delle cosiddette funzioni per nomi di
file.\footnote{Per un elenco esaustivo di queste funzioni vedi
  \textcite[83]{gnu:make}.} Fra tutte ne ricordiamo una:
\begin{lstlisting}
$(wildcard modello)
\end{lstlisting}
Al posto di \verb|modello| bisogna inserire uno schema di nome di file,
generalmente contenente un metacarattere. Per esempio, con
\begin{lstlisting}
$(wildcard *.tex)
\end{lstlisting}
si indica un elenco di tutti i file, presenti nella cartella, con estensione
\verb|.tex|.  Risulta particolarmente utile per ottenere un elenco di file che
hanno tutti uno stesso formato o uno stesso schema nel nome.

Un ultimo strumento importante sono le funzioni per le sostituzioni di
stringhe.%
\footnote{Per un elenco esaustivo di queste funzioni vedi
  \textcite[80]{gnu:make}.}  In particolare la funzione
\begin{lstlisting}
$(patsubst modello,sostituzione,testo)
\end{lstlisting}
permette di sostituire \verb|modello| con \verb|sostituzione| all'interno di
\verb|testo|.  \verb|modello| potrebbe contenere il metacarattere \verb|%| che
rappresenta una qualsiasi sequenza di caratteri e numeri.  Se anche
\verb|sostituzione| contiene \emph{la stessa} sequenza indicata da \verb|%|
allora la sostituzione viene eseguita.  Per esempio
\begin{lstlisting}
$(patsubst %.png,%.eps,img1.png img2.jpg img3.png)
\end{lstlisting}
viene interpretato da \verb|make| come
\begin{verbatim}
img1.eps img3.eps
\end{verbatim}
\verb|img2.jpg| non ha lo stesso schema di \verb|modello|, non finisce cioè per
\verb|.png|, e quindi la sostituzione non avviene, ma le altre due parole
seguono lo schema del modello per cui \verb|img1.png| e \verb|img3.png| vengono
sostituite rispettivamente con \verb|img1.eps| e \verb|img3.eps|.


\section{Un \texttt{Makefile} ancora più complesso}
Il codice~\ref{lst:variabili} è già un \verb|Makefile| abbastanza elaborato e
utile per automatizzare comandi piuttosto lunghi.

Quando in un documento si inseriscono delle immagini, queste devono avere un
formato particolare a seconda che si compili il documento con \LaTeX{}
o con \textsc{PDF}\LaTeX.  In particolare, \LaTeX{}
richiede immagini in formato \textsc{EPS}, \textsc{PDF}\LaTeX{}
accetta immagini in formato \textsc{PDF}, \textsc{JPG} e \textsc{PNG} (vedi
\textcite[105]{pantieri:latex}).  Se si possiede un'immagine in un formato
diverso da quello necessario per l'inserimento nel documento, si deve quindi
procedere alla conversione da un formato all'altro.  Il programma
\verb|ImageMagick| fornisce il comando da terminale \verb|convert|. Con la
sintassi
\begin{verbatim}
convert fileinput fileoutput
\end{verbatim}
si converte il primo file specificato, cioè \verb|fileinput|, nel secondo file,
chiamato in questo caso \verb|fileoutput|.  Per esempio supponiamo di avere
nella sottocartella \verb|Immagini| della cartella in cui si trovano il
documento \LaTeX{}
principale e il \verb|Makefile| tutte le immagini in formato \textsc{PNG}.  Per
compilare in \textsc{DVI} è quindi necessario convertire tutte le immagini.  Per
automatizzare la compilazione e la conversione delle immagini è possibile
scrivere queste regole utilizzando le funzioni apprese nel
paragrafo~\ref{sec:variabili}:
\begin{lstlisting}[caption={\texttt{Makefile} in cui le immagini \textsc{PNG}
vengono convertite in \textsc{EPS} nella compilazione con \LaTeX},label=lst:png-eps]
PRINCIPALE 		= documento
PRINCIPALE_TEX		= $(PRINCIPALE).tex
PRINCIPALE_DVI		= $(PRINCIPALE).dvi
BIBLIOGRAFIA		= bibliografia
BIBLIOGRAFIA_BIB	= $(BIBLIOGRAFIA).bib
TUTTI_LATEX		= $(PRINCIPALE_TEX) \
			  $(BIBLIOGRAFIA_BIB)
CARTELLA_IMG		= Immagini
IMMAGINI_PNG		= $(wildcard $(CARTELLA_IMG)/*.png)
IMMAGINI_EPS		= $(patsubst $(CARTELLA_IMG)/%.eps,\
			  $(CARTELLA_IMG)/%.png, $(IMMAGINI_PNG))

.PHONY: dvi immagini-eps

dvi: $(PRINCIPALE_DVI)

$(PRINCIPALE_DVI): $(TUTTI_LATEX) immagini-eps
	latex $(PRINCIPALE)
	bibtex $(PRINCIPALE)
	makeindex $(PRINCIPALE)
	latex $(PRINCIPALE)
	latex $(PRINCIPALE)

immagini-eps: $(IMMAGINI_EPS)

Immagini/%.eps: Immagini/%.png
	convert $^ $@
\end{lstlisting}
La variabile \verb|IMMAGINI_PNG| contiene l'elenco di tutti i file in formato
\textsc{PNG} presenti nella sottocartella Immagini (il nome della sottocartella
è salvato nella variabile \verb|CARTELLA_IMG| in modo che sarà sufficiente
cambiare solo questo valore per cartelle con nomi differenti).  La variabile
\verb|IMMAGINI_EPS|, invece, contiene l'elenco dei degli stessi file in formato
\textsc{PNG}, ma questa volta con estensione \verb|.eps| (abbiamo utilizzato la
funzione \verb|patsubst| per eseguire la sostituzione).  In questo
\verb|Makefile| è presente una regola con una definizione un po' particolare:
\begin{lstlisting}
Immagini/%.eps: Immagini/%.png
	convert $^ $@
\end{lstlisting}
Il prerequisito di questa regola è un qualsiasi file della sottocartella
\verb|Immagini| con estensione \verb|.png|, l'obiettivo, però, non è un
qualsiasi file della sottocartella \verb|Immagini| con estensione \verb|.eps|,
bensì il file che ha lo stesso nome base del prerequisito ed estensione
\verb|.eps|.\footnote{Il metacarattere \texttt{\%} è stato introdotto nel
  paragrafo~\ref{sec:variabili}.}
Con questa regola verranno dunque generati solo file con estensione \verb|.eps|
e nome base uguale a quello di file in formato \textsc{PNG} presente nella
sottocartella \verb|Immagini|.  Il comando eseguito nella regola è
\begin{lstlisting}
	convert $^ $@
\end{lstlisting}
Le due variabili \verb|$^| e \verb|$@| sono delle variabili dette automatiche%
\footnote{Per un elenco esaustivo di queste variabili vedi
  \textcite[112]{gnu:make}.}
e che hanno un significato speciale: la prima indica tutti i prerequisiti della
regola (in questo caso indica solo il file \textsc{PNG} che ha lo stesso nome
dell'obiettivo), la seconda indica l'obiettivo della regola (in questo caso
l'immagine \textsc{EPS} da generare).  Se volessimo convertire un determinato
file chiamato, supponiamo, \verb|immagine.png| in formato \textsc{EPS} potremmo
utilizzare il comando nel terminale:
\begin{verbatim}
make immagine.eps
\end{verbatim}
Per come è scritta la regola con obiettivo \verb|$(PRINCIPALE_DVI)|
nel \verb|Makefile| del codice~\ref{lst:png-eps} non è però necessario
convertire una a una le immagini quando si vuole generare il documento in
formato \textsc{DVI}.  Infatti, compilando con il comando
\begin{verbatim}
make dvi
\end{verbatim}
il programma \verb|make| si preoccupa di verificare se tutti i prerequisiti
elencati in questa regola sono aggiornati.  Uno di essi è \verb|immagini-eps|
che è un phony target: \verb|immagini-eps| ha il compito di convertire tutte le
immagini in formato \textsc{PNG} presenti nella sottocartella \verb|Immagini|
nel formato \textsc{EPS}, adatto per l'inclusione di immagini nel documento
finale \textsc{DVI}.

Riportiamo ora un \verb|Makefile| ancora più complesso di quelli visti fino ad
adesso. Immaginiamo di avere in una cartella il file \verb|documento.tex| come
file \LaTeX{}
principale, \verb|bibliografia.bib| come raccolta dei riferimenti bibliografici.
Nelle sottocartelle \verb|Capitolo1| e \verb|Capitolo2| sono presenti vari file
\verb|.tex| inclusi in \verb|documento.tex|; nella sottocartella
\verb|Immagini|, invece, sono presenti immagini nel solo formato \textsc{PNG}
che saranno eventualmente convertite nel formato \textsc{EPS} nel caso di
compilazione con \LaTeX.  Un \verb|Makefile| per compilare questo progetto
potrebbe apparire così:
\begin{lstlisting}
PRINCIPALE 		= documento
PRINCIPALE_TEX		= $(PRINCIPALE).tex
PRINCIPALE_DVI		= $(PRINCIPALE).dvi
PRINCIPALE_PDF		= $(PRINCIPALE).pdf
CAPITOLO1_TEX		= $(wildcard Capitolo1/*.tex)
CAPITOLO2_TEX		= $(wildcard Capitolo2/*.tex)
BIBLIOGRAFIA		= bibliografia
BIBLIOGRAFIA_BIB	= $(BIBLIOGRAFIA).bib
TUTTI_LATEX		= $(PRINCIPALE_TEX) \
			  $(BIBLIOGRAFIA_BIB) \
			  $(CAPITOLO1_TEX) \
			  $(CAPITOLO2_TEX)
FILE_CLEAN		= *.aux *.bbl *.blg *.brf *.idx \
			  *.ilg *.ind *.log
FILE_DISTCLEAN		= $(PRINCIPALE_DVI) \
			  $(PRINCIPALE_PDF) \
			  $(IMMAGINI_EPS)
IMMAGINI_PNG		= $(wildcard Immagini/*.png)
IMMAGINI_EPS		= $(patsubst Immagini/%.eps,\
			  Immagini/%.png, $(IMMAGINI_PNG))

.PHONY: dvi pdf distclean clean immagini-eps

dvi: $(PRINCIPALE_DVI)

pdf: $(PRINCIPALE_PDF)

$(PRINCIPALE_DVI): $(TUTTI_LATEX) immagini-eps
	latex $(PRINCIPALE)
	bibtex $(PRINCIPALE)
	makeindex $(PRINCIPALE)
	latex $(PRINCIPALE)
	latex $(PRINCIPALE)

$(PRINCIPALE_PDF): $(TUTTI_LATEX)
	pdflatex $(PRINCIPALE)
	bibtex $(PRINCIPALE)
	makeindex $(PRINCIPALE)
	pdflatex $(PRINCIPALE)
	pdflatex $(PRINCIPALE)

immagini-eps: $(IMMAGINI_EPS)

Immagini/%.eps: Immagini/%.png
	convert $^ $@

distclean: clean
	rm -f $(FILE_DISTCLEAN)

clean:
	rm -f $(FILE_CLEAN)
\end{lstlisting}
Con gli strumenti acquisiti fino a questo punto dovrebbe essere comprensibile
come funziona questo \verb|Makefile|, cosa compie ciascuna regola e quali
comandi bisogna dare nel terminale per compilare il documento.

I \verb|Makefile| dei programmi spesso raggiungono svariate migliaia di righe di
codice, per quanto riguarda un documento \LaTeX{}
piuttosto complesso ne possono bastare anche poche decine.  Si possono comunque
realizzare dei \verb|Makefile| molto più lunghi e avanzati di quelli visti in
questa guida.  In Internet è possibile trovare molta documentazione riguardo ai
\verb|Makefile| (in particolare però per compilare programmi).  Una guida
sicuramente importante è la più volte citata \emph{GNU Make}, di Richard
M. Stallman, Roland McGrath e Paul D. Smith che si può trovare all'indirizzo
\url{http://www.gnu.org/software/make/manual/make.pdf}.

%%% Local Variables:
%%% mode: latex
%%% TeX-master: "../make"
%%% End:
