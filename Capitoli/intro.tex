\cleardoublepage{}
\chapter{Introduzione}
\label{cha:introduzione}

La compilazione di un semplice documento \LaTeX{}
richiede l'esecuzione di uno o due comandi, che generalmente possono essere
facilmente eseguiti all'interno del proprio editor di testo preferito.
All'aumentare delle dimensioni del sorgente si può decidere di suddividere il
sorgente in differenti file di dimensioni più piccole, magari ciascuno contenete
il codice di un intero capitolo, e più facili da gestire.  Anche questa
situazione non costituisce normalmente un grosso problema perché i più diffusi
editor di testo sono in grado di gestire automaticamente progetti di questo
tipo.  L'utilizzo di alcuni pacchetti richiede, però, l'esecuzione di comandi
particolari non accessibili all'interno dell'editor, oppure potrebbe esserci
bisogno di eseguire dei comandi da terminale su file non legati a \LaTeX{}.
Per risolvere alcuni di questi problemi sono stati sviluppati recentemente dei
programmi appositi, come
\texttt{latexmk}\footnote{Per maggiori informazioni su questo programma visita
  il sito \url{http://www.phys.psu.edu/~collins/software/latexmk-jcc/}.}
e
\texttt{arara}.\footnote{Per maggiori informazioni su questo programma visita il
  sito \url{http://cereda.github.com/arara/}.}
In questa guida, invece, ci occuperemo di un programma creato nel 1977 da Stuart
Feldman: \texttt{make}.  Nella voce di Wikipedia in lingua italiana di
\texttt{make} (\url{https://it.wikipedia.org/wiki/Make}) possiamo leggere la
seguente descrizione:
\begin{quoting}
  Il \textbf{make} è un'utility [...] che automatizza il processo di creazione
  di file che dipendono da altri file, risolvendo le dipendenze e invocando
  programmi esterni per il lavoro necessario.
\end{quoting}
\texttt{make} è molto usato nella compilazione dei programmi scritti in C o C++,
specie nell'ambiente del software libero, ma può essere utilmente sfruttato per
automatizzare e semplificare anche la compilazione di complessi documenti
\LaTeX{}.
Rispetto a software come \texttt{arara}, \texttt{make} permette di svolgere, su
richiesta, altre operazioni quali la cancellazione di file temporanei non
necessari e la creazione di un archivio compresso per conservare i file
sorgenti.\footnote{\texttt{arara} permette di eseguire qualsiasi operazione,
  però queste vengono eseguite tutte, ogni volta che viene invocato
  \texttt{arara}, non è possibile scegliere selettivamente quali operazioni
  svolgere.}  Non ci sono limiti alla fantasia.

Di \texttt{make} esistono alcune versioni, le più famose sono \texttt{GNU make},
sviluppata da Richard Stallman e Roland McGrath, e \texttt{BSD make}.  Questa
guida è incentrata su \texttt{GNU make}, il dialetto presente nei sistemi
GNU/Linux e Mac OS X.  Questo programma si utilizza da linea di comando, chi non
dovesse avere familiarità con il terminale può leggere la guida
di~\cite{giacomelli:console}.  Il manuale di riferimento per \texttt{GNU make},
a cui si rimanda per qualsiasi approfondimento, è~\cite{gnu:make}, ma è
orientato soprattutto a programmatori che sviluppano software nel linguaggio C,
la presente guida vuole essere invece una semplice introduzione a \texttt{make}
per gli utenti di \LaTeX{}.

Non si vuole convincere nessuno che \texttt{make} sia il programma migliore in
assoluto, perché sicuramente \emph{non} lo è, ha i suoi pregi e anche qualche
difetto.  Qui verranno solo illustrate le sue funzioni e potenzialità fornendo
anche degli esempi d'uso, starà poi al lettore decidere se \texttt{make} si
adatta alle proprie necessità.

Sono ben accetti suggerimenti per migliorare la guida, segnalazioni di errori e
richieste di chiarimenti.

\begin{flushright}
\begin{minipage}{0.6\textwidth}\centering
Mosè Giordano \\
\texttt{giordano dot mose at libero dot it}
\end{minipage}
\end{flushright}

%%% Local Variables:
%%% mode: latex
%%% TeX-master: "../make"
%%% End:
