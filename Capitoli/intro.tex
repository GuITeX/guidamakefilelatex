\cleardoublepage
\chapter{Introduzione}
\label{cha:introduzione}

\section{Cos'è \texttt{make}}
\label{sec:cose-make}

La compilazione di un semplice documento \LaTeX{}
richiede l'esecuzione di uno o due comandi, che generalmente possono essere
facilmente eseguiti all'interno del proprio editor di testo preferito.
All'aumentare delle dimensioni del sorgente si può decidere di suddividere il
sorgente in differenti file di dimensioni più piccole, magari ciascuno contenete
il codice di un intero capitolo, e più facili da gestire.  Anche questa
situazione non costituisce normalmente un grosso problema perché i più diffusi
editor di testo sono in grado di gestire automaticamente progetti di questo
tipo.  L'utilizzo di alcuni pacchetti richiede, però, l'esecuzione di comandi
particolari non accessibili all'interno dell'editor, oppure potrebbe esserci
bisogno di esegui dei comandi su file completamente estranei all'infrastruttura
\LaTeX{}.
Per risolvere alcuni di questi problemi sono stati sviluppati recentemente dei
programmi appositi, come
\texttt{latexmk}\footnote{Per maggiori informazioni su questo programma visita
  il sito \url{http://www.phys.psu.edu/~collins/software/latexmk-jcc/}.}
e
\texttt{arara}.\footnote{Per maggiori informazioni su questo programma visita il
  sito \url{http://cereda.github.com/arara/}.}
In questa guida, invece, ci occuperemo di un programma creato nel 1977 da Stuart
Feldman: \texttt{make}.  Nella voce di Wikipedia in lingua italiana di
\texttt{make} (\url{https://it.wikipedia.org/wiki/Make}) possiamo leggere la
seguente descrizione:
\begin{quoting}
  Il \textbf{make} è un'utility [...] che automatizza il processo di creazione
  di file che dipendono da altri file, risolvendo le dipendenze e invocando
  programmi esterni per il lavoro necessario.
\end{quoting}
\texttt{make} è molto usato nella compilazione dei programmi scritti in C o C++,
specie nell'ambiente del software libero, ma può essere utilmente sfruttato per
automatizzare e semplificare anche la compilazione di complessi documenti
\LaTeX{}.
Rispetto a software come \texttt{arara}, \texttt{make} permette di svolgere, su
richiesta, altre operazioni quali la cancellazione di file temporanei non
necessari e la creazione di un archivio compresso per conservare i file
sorgenti.\footnote{\texttt{arara} permette di eseguire qualsiasi operazione,
  però queste vengono eseguite tutte, ogni volta che viene invocato
  \texttt{arara}, non è possibile scegliere selettivamente quali operazioni
  svolgere.}  Non ci sono limiti alla fantasia.

Il programma \texttt{make} non fa altro che leggere le istruzioni presenti in un
file di testo, chiamato necessariamente
\texttt{Makefile}\footnote{In realtà potrebbe avere anche altri nomi, questo è
  quello consigliato e riconosciuto in maniera predefinita, per ulteriori
  informazioni vedi~\textcite[12]{gnu:make}.}
e scritto con una particolare sintassi.  I \texttt{Makefile} sono dei file di
testo puro, per scriverli c'è bisogno quindi di un normale editor di testo come
Gedit o Kate per GNU/Linux, TextEdit per Mac OS X.

Di \texttt{make} esistono alcune versioni, le più famose sono \texttt{GNU make},
sviluppata da Richard Stallman e Roland McGrath, e \texttt{BSD make}.  In questa
guida verrà spiegato come scrivere \texttt{Makefile} elementari per
\texttt{GNU make}, il dialetto presente nei sistemi GNU/Linux e Mac OS X.
Questo programma si utilizza da linea di comando, chi non dovesse avere
familiarità con il terminale può leggere la guida
di~\textcite{giacomelli:console}.  La guida di riferimento per
\texttt{GNU make}, a cui rimando per qualsiasi approfondimento,
è~\textcite{gnu:make}, ma è orientata soprattutto a programmatori che sviluppano
software C, la presente guida vuole essere invece una semplice introduzione a
\texttt{make} per gli utenti di \LaTeX{}.

Con questa guida non voglio convincere nessuno che \texttt{make} sia il
programma migliore in assoluto, perché sicuramente \emph{non} lo è, ha i suoi
pregi e anche qualche difetto.  Qui voglio solo illustrare le sue funzioni,
starà poi al lettore decidere se \texttt{make} si adatta alle proprie necessità.

\section{Installazione di \texttt{make}}
\label{sec:installazione}

Per ottenere \texttt{make} in ambiente GNU/Linux, se non già presente nel
sistema bisogna installare il pacchetto omonimo utilizzando il gestore pacchetti
della propria distribuzione.  Per esempio, per Debian e sistemi derivati (come
Ubuntu) bisogna dare il comando
\begin{verbatim}
# apt-get install make
\end{verbatim}
con i diritti di amministratore.  In Fedora e derivate, invece, bisogna eseguire
il comando
\begin{verbatim}
# yum install make
\end{verbatim}

Sui sistemi Mac OS X, \texttt{make} dovrebbe essere preinstallato.  Per
verificarlo si può dare il comando nel Terminale
\begin{verbatim}
$ make --version
\end{verbatim} % $
che restituisce la versione del programma installata nel sistema.  Se il
programma non è presente verrà mostrato un messaggio di errore.

\textbf{Nota}: da qui in poi tutti i comandi da eseguire nel terminale devono
essere lanciati trovandosi nella stessa cartella in cui è presente il
\texttt{Makefile} che si desidera utilizzare, se non diversamente specificato.

%%% Local Variables:
%%% mode: latex
%%% TeX-master: "../make"
%%% End:
