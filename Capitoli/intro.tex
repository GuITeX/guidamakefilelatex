\cleardoublepage
\chapter{Introduzione}
\label{cha:introduzione}

Quando si scrive un documento \LaTeX{}
molto elaborato, potrebbe diventare necessario dare numerosi comandi in fase di
compilazione, comandi che a volte risultano molto lunghi o di difficile
memorizzazione.  In questo caso viene in aiuto la utility dei sistemi UNIX
\verb|make| che si utilizza da linea di comando.  Nella voce di Wikipedia in
lingua italiana di \verb|make| (\url{http://it.wikipedia.org/wiki/Make})
possiamo leggere la seguente descrizione:
\begin{quoting}
  Il \textbf{make} è un'utility [...] che automatizza il processo di creazione
  di file che dipendono da altri file, risolvendo le dipendenze e invocando
  programmi esterni per il lavoro necessario.
\end{quoting}
\verb|make| è molto usata nella compilazione dei programmi, specie nell'ambiente
del software libero, ma può essere utilmente sfruttata per automatizzare e
semplificare anche la compilazione di sorgenti \LaTeX{}
e anche altre operazioni quali la cancellazione di file temporanei non necessari
e la creazione di un archivio compresso per conservare i file sorgenti.  Non ci
sono limiti alla fantasia.

Per compilare un sorgente, \verb|make| non fa altro che leggere le istruzioni
presenti in un file di testo, chiamato necessariamente
\verb|Makefile|\footnote{In realtà potrebbe avere anche altri nomi, questo è
  quello consigliato, vedi \textcite[12]{gnu:make}.}
e scritto con una particolare sintassi.  I \verb|Makefile| sono dei file di
testo puro, per scriverli c'è bisogno quindi di un normale editor di testo come
Gedit o Kate per GNU/Linux, TextEdit per Mac OS X.

Di \verb|make| esistono alcune versioni, le più famose sono \verb|GNU make| e
\verb|BSD make|.  In questa guida verrà spiegato come scrivere \verb|Makefile|
elementari per \verb|GNU make|, cioè il dialetto presente nei sistemi GNU/Linux
e Mac OS X.

\textbf{Nota}: da qui in poi (tranne nel Paragrafo \ref{sec:installazione}),
tutti i comandi da eseguire nel terminale devono essere lanciati trovandosi
nella stessa cartella in cui è presente il \verb|Makefile| che si desidera
utilizzare, se non diversamente specificato.


\section{Installazione di \texttt{make}}
\label{sec:installazione}

Per ottenere \verb|make| in ambiente GNU/Linux, se non già presente nel sistema
bisogna installare il pacchetto omonimo utilizzando il gestore pacchetti della
propria distribuzione.  Per esempio, per Debian e sistemi derivati (come Ubuntu)
bisogna dare il comando
\begin{verbatim}
apt-get install make
\end{verbatim}
con i diritti di root.  In Fedora e derivate, invece, bisogna eseguire il
comando
\begin{verbatim}
yum install make
\end{verbatim}

Sui sistemi Mac OS X, \verb|make| dovrebbe essere preinstallato.  Per
verificarlo si può dare il comando nel Terminale
\begin{verbatim}
make --version
\end{verbatim}
che restituisce la versione del programma installata nel sistema.  Se il
programma non è presente verrà mostrato un messaggio di errore.

%%% Local Variables:
%%% mode: latex
%%% TeX-master: "../make"
%%% End:
